% -------------------------------------------------------
% Daten für die Arbeit
% Wenn hier alles korrekt eingetragen wurde, wird das Titelblatt
% automatisch generiert. D.h. die Datei titelblatt.tex muss nicht mehr
% angepasst werden.

\newcommand{\hsmasprache}{de} % de oder en für Deutsch oder Englisch
% Für korrekt sortierte Literatureinträge, noch preambel.tex anpassen
% und zwar bei \usepackage[main=ngerman, english]{babel},
% \usepackage[pagebackref=false,german]{hyperref}
% und \usepackage[autostyle=true,german=quotes]{csquotes}

% Titel der Arbeit auf Deutsch
\newcommand{\hsmatitelde}{Embedded Frontend-Entwicklung mit .Net Core und Blazor}

% Titel der Arbeit auf Englisch
\newcommand{\hsmatitelen}{Embedded Frontend-Development with .Net Core and Blazor}

% Weitere Informationen zur Arbeit
\newcommand{\hsmaort}{Offenburg}    % Ort
\newcommand{\hsmaautorvname}{William} % Vorname(n)
\newcommand{\hsmaautornname}{Mendat} % Nachname(n)
\newcommand{\hsmadatum}{28.02.2022} % Datum der Abgabe
\newcommand{\hsmajahr}{2022} % Jahr der Abgabe
\newcommand{\hsmafirma}{Junker Technologies} % Firma bei der die Arbeit durchgeführt wurde
\newcommand{\hsmabetreuer}{Prof. Dr.-Ing. Daniel Fischer, Hochschule Offenburg} % Betreuer an der
\newcommand{\hsmazweitkorrektor}{M.Sc. Adrian Junker, Junker Technologies} % Betreuer im
\newcommand{\hsmafakultaet}{EMI} % Fakultät
\newcommand{\hsmastudiengang}{AI} % Studiengangsabkürzung.
% Diese wird in titelblatt.tex definiert. Bisher AI, EI, MK und INFM. Bitte ergänzen.

% Zustimmung zur Veröffentlichung
\setboolean{hsmapublizieren}{true}   % Einer Veröffentlichung wird zugestimmt
\setboolean{hsmasperrvermerk}{false} % Die Arbeit hat keinen Sperrvermerk

% -------------------------------------------------------
% Abstract

% Kurze (maximal halbseitige) Beschreibung, worum es in der Arbeit geht auf Deutsch
\newcommand{\hsmaabstractde}{In dieser Abschlussarbeit wird eine .Net Core Blazor Anwendung
für ein Non-Deeply Embedded System implementiert. Dabei werden verschiedene Ansätze auf dem
Raspberry Pi implementiert, um einen aussagekräftigen
Vergleich zwischen den hier Verschiedenen Technologien zu schaffen. Ein Teil dieser Ausarbeitung
besteht darin, eine Laufzeitanalyse zu erstellen.

Die Arbeit gibt zunächst einen Einblick in den momentanen Stand der Technik, wie
bislang eine GUI für ein Non-Deeply Embedded System implementiert wurde, um dann zu
veranschaulichen, wie
dass gleiche Verhalten mit .Net Core und Blazor widergespiegelt werden kann.
Am Ende dieser Arbeit wird ein aussagekräftiges Fazit darüber abgegeben, ob die
Technologie Blazor für die Frontend-Entwicklung im Non-Deeply Embedded Bereich zu gebrauchen ist.}

% Kurze (maximal halbseitige) Beschreibung, worum es in der Arbeit geht auf Englisch

\newcommand{\hsmaabstracten}{In this thesis we implement a .Net Core Blazor Application for a non-deeply embedded system.
Multiple approaches for a Raspberry Pi are implemented to create meaningful comparison between the different technologies. One part of this thesis is the creation of a run time analysis.

This thesis will first give an overview of the currently used technologies and how GUI's for
non-deeply embedded system are created, before then showing how the same behaviour can be implemented in .Net Core using Blazor.

At the end of this thesis we conclude if the Blazor technology is suitable for use in front-end development for non-deeply embedded systems.}
