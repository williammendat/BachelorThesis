% -------------------------------------------------------
% Daten für die Arbeit
% Wenn hier alles korrekt eingetragen wurde, wird das Titelblatt
% automatisch generiert. D.h. die Datei titelblatt.tex muss nicht mehr
% angepasst werden.

\newcommand{\hsmasprache}{de} % de oder en für Deutsch oder Englisch
% Für korrekt sortierte Literatureinträge, noch preambel.tex anpassen
% und zwar bei \usepackage[main=ngerman, english]{babel},
% \usepackage[pagebackref=false,german]{hyperref}
% und \usepackage[autostyle=true,german=quotes]{csquotes}

% Titel der Arbeit auf Deutsch
\newcommand{\hsmatitelde}{Embedded Frontend-Entwicklung mit .Net Core und Blazor}

% Titel der Arbeit auf Englisch
\newcommand{\hsmatitelen}{Embedded Frontend-Development with .Net Core and Blazor}

% Weitere Informationen zur Arbeit
\newcommand{\hsmaort}{Offenburg}    % Ort
\newcommand{\hsmaautorvname}{William} % Vorname(n)
\newcommand{\hsmaautornname}{Mendat} % Nachname(n)
\newcommand{\hsmadatum}{28.02.2022} % Datum der Abgabe
\newcommand{\hsmajahr}{2022} % Jahr der Abgabe
\newcommand{\hsmafirma}{Junker Technologies} % Firma bei der die Arbeit durchgeführt wurde
\newcommand{\hsmabetreuer}{Prof. Dr.-Ing. Daniel Fischer, Hochschule Offenburg} % Betreuer an der
\newcommand{\hsmazweitkorrektor}{M. Sc. Adrian Junker, Junker Technologies} % Betreuer im
\newcommand{\hsmafakultaet}{EMI} % Fakultät
\newcommand{\hsmastudiengang}{AI} % Studiengangsabkürzung.
% Diese wird in titelblatt.tex definiert. Bisher AI, EI, MK und INFM. Bitte ergänzen.

% Zustimmung zur Veröffentlichung
\setboolean{hsmapublizieren}{true}   % Einer Veröffentlichung wird zugestimmt
\setboolean{hsmasperrvermerk}{false} % Die Arbeit hat keinen Sperrvermerk

% -------------------------------------------------------
% Abstract

% Kurze (maximal halbseitige) Beschreibung, worum es in der Arbeit geht auf Deutsch
\newcommand{\hsmaabstractde}{In dieser Abschlussarbeit wird eine Net Core Blazer Anwendung implementiert, um
somit einen Ersatz für die derzeit GUI-Entwicklung auf dem Raspberry Pi mit Qt
zu kreieren. Dabei werden verschiedene Ansätze implementiert, sowohl auf dem
Raspberry Pi als auch auf einem externen Server, um einen aussagekräftigen
Vergleich zwischen den hier verschiedenen Technologien zu schaffen. Ein großer
Teil dieser Ausarbeitung besteht darin, eine Laufzeitanalyse zu erstellen.

Die Arbeit gibt zunächst einen Einblick in den momentanen Stand der Technik, wie
bislang mit einem Raspberry Pi gearbeitet wurde, um dann zu veranschaulichen, wie
dass gleiche Verhalten mit Net Core und Blazer widergespiegelt werden kann.
Am Ende dieser Arbeit wird ein aussagekräftiges Fazit darüber abgegeben, ob die
Technologie Blazor für die Frontendentwicklung im Embedded Bereich zu gebrauchen ist.}

% Kurze (maximal halbseitige) Beschreibung, worum es in der Arbeit geht auf Englisch

\newcommand{\hsmaabstracten}{Englische Version von Lorem ipsum dolor sit amet, consetetur sadipscing elitr, sed diam nonumy eirmod tempor invidunt ut labore et dolore magna aliquyam erat, sed diam voluptua. At vero eos et accusam et justo duo dolores et ea rebum. Stet clita kasd gubergren, no sea takimata sanctus est Lorem ipsum dolor sit amet. Lorem ipsum dolor sit amet, consetetur sadipscing elitr, sed diam nonumy eirmod tempor invidunt ut labore et dolore magna aliquyam erat, sed diam voluptua. At vero eos et accusam et justo duo dolores et ea rebum. Stet clita kasd gubergren, no sea takimata sanctus est Lorem ipsum dolor sit amet.}
