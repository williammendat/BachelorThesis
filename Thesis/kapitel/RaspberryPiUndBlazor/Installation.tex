\section{Installation}
\label{sec:installation}
Da in der vorherigen Sektion \emph{\nameref{sec:entwicklungsumgebung}}, Visual Studio Code eingerichtet wurde und mit dem Raspberry Pi
Remote verbunden wurde, kann nun auch das integrierte Terminal von Visual Studio Code für die
Installation von .Net Core beziehungsweise .Net verwendet werden.
\newline
\newline
Zuallererst muss überprüft werden, ob es sich bei dem Raspberry Pi um die 32-bit oder die 64-bit
version handelt. Dies kann mit dem befehl \emph{uname -a} überprüft werden. Dabei können folgende
zwei ergebnisse erfolgen:
\begin{zitat}
    Linux raspberrypi 4.19.97-v7l+ #1294 SMP Thu Jan 30 13:23:13 GMT 2020 armv7l GNU/Linux
    \newline
    Linux raspberrypi 5.10.60-v8+ #1291 SMP Thu Jan 30 13:21:14 GMT 2020 aarch64 GNU/Linux
\end{zitat}
Sollte das Resultat nun \emph{armv7l GNU/Linux} anzeigen, dann handelt es sich bei dem Raspberry
Pi um die 32-bit version, sollte aber \emph{aarch64 GNU/Linux} angezeigt werden, dann ist es die
64-bit version.
\newline
\newline
Nun kann entweder auf der offiziellen Microsoft seite, die richtige SDK heruntergeladen werden,
oder mithilfe von \emph{wget} die SDK vom Terminal heruntergeladen werden. Nachdem der Download
beendet ist, können mit den folgenden befehlen, die Installation beendet werden:
\begin{zitat}
    mkdir -p \$HOME/dotnet
    \newline
    tar zxf dotnet-sdk-6.0.100-rc.1.21458.32-linux-arm.tar.gz -C \$HOME/dotnet
    \newline
    export DOTNET\_ROOT=\$HOME/dotnet
    \newline
    export PATH=\$PATH: \$HOME/dotnet
\end{zitat}

Diese Befehle erstellen einen neuen Ordner namens \emph{dotnet}, entpacken die SDK und packen den
Inhalt in den neu erstellten Ordner und setzten zu guter letzt noch die \emph{Path-Variable} zu
dem dotnet Ordner.
\newline
\newline
Mit dem Befehl \emph{dotnet --info} kann anschließend überprüft werden, ob die Installation
erfolgreich war.