\subsection{Microsoft Iot}
\label{subsec:MicrosoftIot}
Damit mit dem Raspberry Pi kommuniziert werden kann, können Bibliotheken verwendet werden.
Microsoft bietet eine Iot Bibliothek an, mit der dann die Sensoren auf dem Gerät oder die LED´s auf
dem Gerät angesteuert werden können. Die Bibliothek kann in einem Projekt entweder durch den
\emph{Nuget Packager} eingebunden werden, oder indem die \emph{PackageReference} wie folgt, in
der Solution Datei konfiguriert wird:

\begin{lstlisting}[language={[Sharp]C}, caption=Iot Nuget Package,
    label=lst:IotNugetPackage]
    <ItemGroup>
    <PackageReference Include="Iot.Device.Bindings" Version="1.5.0-*" />
  </ItemGroup>
\end{lstlisting}

Nun kann der Programmier zum Beispiel \emph{using Iot.Device.SenseHat} in den Code einbinden um
auf die Sensoren des Sensehat´s zuzugreifen. Damit veranschaulicht wird, wie auf die
Daten zugegrifen werden kann, wird ein Beispielhaftes Konsolen Programm demonstriert:

\begin{lstlisting}[language={[Sharp]C}, caption=SenseHat Beispiel Programm,
    label=lst:SenseHatBeispielProgramm]
    public class Program
    {
        static void Main(string[] args)
        {
            using SenseHat _senseHat = new();

            Console.WriteLine($"Temperatur: {_senseHat.Temperature.DegreesCelsius:0.#}\u00B0C");
            Console.WriteLine($"Temperatur 2: {_senseHat.Temperature2.DegreesCelsius:0.#}\u00B0C");
            Console.WriteLine($"Luftdruck: {_senseHat.Pressure.Hectopascals:0.##} hPa");
            Console.WriteLine($"Luftfeuchtigkeit: {_senseHat.Humidity.Percent:0.#}%");
        }
    }
\end{lstlisting}

Der Output der bei dem Beispiel erzeugt wird, könnte so aussehen:

\begin{zitat}
    Temperatur: 38,4°C
    \newline
    Temperatur 2: 38,5°C
    \newline
    Luftdruck: 984,03 hPa
    \newline
    Luftfeuchtigkeit: 31\%
\end{zitat}