\subsection{Microsoft IoT}
\label{subsec:MicrosoftIot}
Damit mit dem Raspberry Pi kommuniziert werden kann, können Bibliotheken verwendet werden.
Microsoft bietet eine IoT Bibliothek an, mit der dann die Sensoren auf dem Gerät oder die LEDs auf
dem Gerät angesteuert werden können. Die Bibliothek kann wie folgt in das Projekt eingebunden
werden:

\begin{lstlisting}[language={[Sharp]C}, caption=IoT NuGet Package,
    label=lst:IotNugetPackage]
    <ItemGroup>
    <PackageReference Include="Iot.Device.Bindings" Version="1.5.0-*" />
  </ItemGroup>
\end{lstlisting}

Das folgende beispielhafte Konsolenprogramm demonstriert, wie auf die Daten zugegriffen werden kann.

\begin{lstlisting}[language={[Sharp]C}, caption=SenseHat Beispielprogramm,
    label=lst:SenseHatBeispielProgramm]
    public class Program
    {
        static void Main(string[] args)
        {
            using SenseHat _senseHat = new();

            Console.WriteLine($"Temperatur: {_senseHat.Temperature.DegreesCelsius:0.#}\u00B0C");
            Console.WriteLine($"Temperatur 2: {_senseHat.Temperature2.DegreesCelsius:0.#}\u00B0C");
            Console.WriteLine($"Luftdruck: {_senseHat.Pressure.Hectopascals:0.##} hPa");
            Console.WriteLine($"Luftfeuchtigkeit: {_senseHat.Humidity.Percent:0.#}%");
        }
    }
\end{lstlisting}

Der Output durch obiges Programm sieht wie folgt aus:

\begin{zitat}
    Temperatur: 38,4°C
    \newline
    Temperatur 2: 38,5°C
    \newline
    Luftdruck: 984,03 hPa
    \newline
    Luftfeuchtigkeit: 31\%
\end{zitat}