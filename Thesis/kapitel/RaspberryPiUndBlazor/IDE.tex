\section{Entwicklungsumgebung}
\label{sec:entwicklungsumgebung}
Die Entwicklungsumgebung, die in dieser Thesis eingesetzt wird ist Visual Studio Code. Dabei
wurde Visual Studio Code, da viele Community Plugins zur Verfügung stehen,
die das Entwickeln auf dem Raspberry Pi unterstützten. Zum einen ist die Möglichkeit gegeben, sich
Remote mit den Raspberry Pi zu verbinden und zum anderen
existiert noch die Möglichkeit des Remote Debuggings. Beim Remote Debugging wird der Code auf dem
Host Rechner entwickelt und auf dem Target ausgeführt.
\newline
\newline
Im Zuge dieser Thesis wird sich Remote mit dem Raspberry Pi verbunden, um dann auf dem Target zu
programmieren. Dafür muss die Extension \emph{Remote - SSH} von Microsoft in Visual Studio Code
installiert werden. Nun kann mit der Tastenkombination \emph{Strg - Shift - p} ein neuer Dialog
geöffnet werden, in welchem dann die Option \emph{Remote-SSH Connect to host} ausgewählt werden
kann. Nachdem \emph{Remote-SSH Connect to host} ausgewählt wurde, muss der Befehl
\emph{<benutzernamen>@<ip adresse>} eingegeben werden. Zuletzt muss lediglich noch das Password
eingegeben werden.