\subsection{Erstellen des Projektes}
\label{subsec:erstellenProject}
Um eine Blazor Anwendung zu erstellen, gibt es zwei Möglichkeiten. Zum einem vom \emph{Scratch}
aus, aus einer einfachen Konsolen Anwendung, um den notwendigen Code zu implementiert, oder zum
anderen mit hilfe eines Templates. Microsoft bietet einige Templates für verschiedenste
Anwendungen an, die alle samt mit der Installation von .Net kommen.
\newline
\newline
Um nun die Anwendung mithilfe des Templates zu erstellen, muss der folgende Befehl in dem Teminal
eingegeben werden:

\begin{zitat}
    dotnet new blazorserver -o MyApp --no-https
\end{zitat}

Dieser Befehl erstellt eine neue Blazor Server Anwendung, mit dem Namen \emph{MyApp} und
konfiguriert die Anwendung ohne das Https Protokol. Der name sowie, dass kein Https konfiguriert
wird, sind optionale Parameter, die nicht mit angegeben werden müssen. Nachdem die Anwendung 
erfolgreich Installiert wurde, muss noch die Codezeile \emph{webBuilder.UseUrls("Http://*:5000");
} in der \emph{Program.cs} angegeben werden, um die Anwendung für alle Geräte im LAN verfügbar
ist. Der Code in der \emph{Program.cs} sieht dann folgendermaßen aus:

\begin{lstlisting}[language={[Sharp]C}, caption=Program.cs Code,
    label=lst:programCsCode]
    public class Program
    {
        // Main

        public static IHostBuilder CreateHostBuilder(string[] args) =>
            Host.CreateDefaultBuilder(args)
                .ConfigureWebHostDefaults(webBuilder =>
                {
                    webBuilder.UseStartup<Startup>();
                    webBuilder.UseUrls("Http://*:5000"); // <----
                });
    }
\end{lstlisting}

Mit dem Befehl \emph{dotnet run} kann die Anwendung dann gestartet werden. Sobald das Programm
gestartet ist, kann mit dem Link \emph{http://<ip>:5000} die Seite erreicht werden. Sollte alles
geklappt haben sollte nun die Seite, mit den schon vom Template gegebenen Komponenten, wie folgt
angezeigt werden:

\begin{figure}[h]
    \centering
    \includegraphics[width=\textwidth, center]{BlazorRasp/ServerTemplate}
    \caption[Blazor Server Template]{Blazor Server Template}
    \label{img:BlazorServerTemplate}
\end{figure}