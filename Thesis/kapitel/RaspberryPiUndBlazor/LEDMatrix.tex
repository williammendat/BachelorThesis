\subsection{LED-Matrix Ansteuern}
\label{subsec:ledMatrix}
In dieser Sektion soll nun eine \emph{8x8 Button-Matrix} erstellt werden, die die \emph{8x8
LED-Matrix} auf dem SenseHat des Raspberry Pi´s repräsentieren soll. Dabei soll der vorhandene
Joystick auf dem SenseHat, dazu genutzt werden um über die Button-Matrix zu navigieren und die
LED´s zu platzieren.
\newline
\newline
Dadurch, das Blazor die \emph{Razor-Syntax} verwendet, kann jegliche C\# kontrollstrucktur im
\emph{Html-Markup} verwendet werden. So kann dies auch wie folgt genutzt werden, um die \emph{8x8
Button-Matrix} zu erzeugen:

\begin{lstlisting}[language={[Sharp]C}, caption=Button-Matrix,
    label=lst:ButtonMatrix]
<div class="divGrid">
    @for (var y = 0; y < LengthY; y++)
    {
        @for (var x = 0; x < LengthX; x++)
        {
            int copyY = y;
            int copyX = x;
            <Button class="buttonBox @classes[copyY,copyX]"/>
        }
    }
</div>
\end{lstlisting}

Das \emph{@classes} element, ist ein pseudo Zwei-Dimensional Array aus strings, mit der dann
Css-Klassen hinzugefügt und wieder gelöscht werden sollen. Pseudo Zwei-Dimensionales Array
deswegen, weil diese in C\# nur statisch erxistieren, aufgrund von inperformance.
\newline
\newline
Um zu ermitteln, ob und wenn ja welcher Joystick button betätigt wurde, soll auch hier wieder ein
\emph{Timer} zum einsatzt kommen, der jedoch diesmal alle 15 Millisekunden getriggert wird.

\begin{lstlisting}[language={[Sharp]C}, caption=Timer: ButtonTimer,
    label=lst:ButtonTimer]
    private void StartTimer()
    {
        // More Code

        _setButtonTimer = new(15);
        _setButtonTimer.Elapsed += WriteStateToChannel;
        _setButtonTimer.Enabled = true;
    }


    private async void WriteStateToChannel(Object source, System.Timers.ElapsedEventArgs e)
    {
        if((ticks - lastTicks) > 9){
            _senseHat.ReadJoystickState();
            if(_senseHat.HoldingButton){
                await _stateChannel.Writer.WriteAsync(JoystickState.Holding);
            }
            else if(_senseHat.HoldingUp){
                await _stateChannel.Writer.WriteAsync(JoystickState.Up);
            }
            else if(_senseHat.HoldingDown){
                await _stateChannel.Writer.WriteAsync(JoystickState.Down);
            }
            else if(_senseHat.HoldingLeft){
                await _stateChannel.Writer.WriteAsync(JoystickState.Left);
            }
            else if(_senseHat.HoldingRight){
                await _stateChannel.Writer.WriteAsync(JoystickState.Right);
            }
            lastTicks = ticks;
        }
        ticks++;
    }
\end{lstlisting}

Bei \emph{JoystickState} handelt es sich um ein Enum, welches den \emph{State} des Joysticks
repräsentieren soll. Wie zu sehen ist, wird der aktuelle \emph{State} des Joysticks gelesen um
dann das Ergebnis in einen \emph{Channel} zu schreiben. Für diesen Timer, wurde bewust das
Konzept von \emph{Channels} genutzt, anstatt die Berechnungen und die State-Änderung direkt im
Timer zu Implementieren, da der \emph{Timer} alle 15 Millisekunden ausgelöst wird und dadurch
vermieden werden sollte, dass das Programm im Timer hängen bleibt.
\newline
\newline
Für die Verarbeitung des \emph{States}, wird in \emph{OnInitializedAsync} ein Task gestartet, der
in einer \emph{endless-loop} läuft und darauf warte, bis etwas in den \emph{Channel} hinzugefügt
wurde. Sobald sich der \emph{State} geändert hat, passiert abhängig von dem Event, wie folgend zu
sehen sein wird, das entweder die Position berechnet wird, oder die LED und der Button an dieser
Position, die Farbe wechselt.

\begin{lstlisting}[language={[Sharp]C}, caption=Task zum Verarbeiten des States,
    label=lst:StateTask]
        _setButtonTask = Task.Run(async () =>
        {
            while (true)
            {
                if(cancellationToken.IsCancellationRequested)
                    cancellationToken.ThrowIfCancellationRequested();
                var state = await _stateChannel.Reader.ReadAsync();
                int x = 0;
                int y = 0;
                if(state == JoystickState.Holding){
                    SetButtonBackground(_currentX, _currentY);
                }
                else if(state == JoystickState.Up){
                    y--;
                }
                else if(state == JoystickState.Down){
                    y++;
                }
                else if(state == JoystickState.Left){
                    x--;
                }else if(state == JoystickState.Right){
                    x++;
                }
                SetPositions(x, y);
                RemoveButtonBorder(_previousX, _previousY);
                SetButtonBorder(_currentX, _currentY);
                await InvokeAsync(StateHasChanged);
            }
        });

    private void SetButtonBackground(int x, int y)
    {
        if (classes[y, x].Contains(" setBackground"))
        {
            _senseHat.SetPixel(x, y, Color.Blue);
            classes[y, x] = classes[y, x].Replace(" setBackground", string.Empty);
        }
        else
        {
            _senseHat.SetPixel(x, y, Color.Red);
            classes[y, x] += " setBackground";
        }
    }
\end{lstlisting}

Wie zu sehen ist, wird, sollte der Joystick gedrückt werden, die LED und der Button je
nach vorherigem Zustand entweder Blau oder Rot. In den anderen Fällen, werden lediglich die neuen
Positionen berechnet und ein Border wird gesetzt, um zu signalisieren an welcher Positon auf der
Matrix, der momentane Fokus ist.
\newline
\newline
Die komplette Anwendung sieht dann wie folgt aus:

\begin{figure}[h]
    \centering
    \includegraphics[width=\textwidth, center]{BlazorRasp/Demo}
    \caption[Blazor Demo]{Blazor Demo}
    \label{img:BlazorDatenAnzeigen}
\end{figure}

Da nur Technisch relevante Code abschnitte in diesem Kapitel präsentiert wurden, ist der
komplette Code im Anhang \ref{lst:DemoCode} zu finden.