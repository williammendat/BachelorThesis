\section{Verwendete Software}
\label{sec:verwendeteSoftware}
Im Rahmen dieser Thesis werden die folgenden Softwaretools zur Entwicklung eingesetzt:
\begin{itemize}
    \item Um eine Visualisierung des Images \emph{Raspbian Buster with desktop} von dem Raspberry
    Pi zu erhalten, wird die Windows Desktop Anwendung \emph{VNC Viewer} verwendet. Dadurch ist
    die Möglichkeit gegeben, bequem und einfach den Raspberry Pi remote zu bedienen.
    \item Für die Demonstrierung des Kapitels \emph{\nameref{chap:standTechnik}} wird eine
    beispielhafte grafische Oberfläche mithilfe des Qt-Creators implementiert.
    \item Den Zugriff auf die Funktionalitäten des \emph{RPI SENSE HAT Shield} wird mittels der
    \emph{RTIMUlib} Bibliothek realisiert.
    \item Die Programmiersprachen dieser Arbeit werden sich hauptsächlich aus \emph{C++} und
    \emph{C\#} beziehungsweise .Net Core zusammensetzen.
    \item Um den Zugriff auf die Funktionalitäten des \emph{RPI SENSE HAT Shield} mittels .Net
    Core zu gewährleisten, wird die Iot Bibliothek von Microsoft verwendet.
    \item Die Implementierung der Anwendung mit Blazor wird dann letztendlich mittels der
    kostenlosen IDE \emph{Visual Studio Code} realisiert.
    \item Um zwischen dem Host Rechner und dem Raspberry Pi Dokumente und Ordner auszutauschen,
    wird die Desktop-Applikation \emph{WinSCP} verwendet.
\end{itemize}

Die oben vorgestellten Tools und Programmiersprachen wurden nicht explizit vorgegeben, sind
dementsprechend auch selbst ausgesucht und sind zu Beginn dieser Thesis auch schon alle komplett
eingerichtet.