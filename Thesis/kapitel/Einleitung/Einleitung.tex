\chapter{Einleitung}
\label{chap:einleitung}
Die Menschheit ist heutzutage darauf fokussiert, die komplette Welt zu digitalisieren. Dabei
existiert ein Grundsatz:
\begin{zitat}
    Alles, was digitalisiert werden kann, wird digitalisiert \cite{digitalisierteWelt}
\end{zitat}

Um dies zu realisieren ist es vonnöten überall Hardware und Software zu verbinden. Sei es nun das
Handy, mit welchem durch nur einen klick die Bankdaten angezeigt werden können, oder ein
selbstfahrendes Auto, welches einen Anwender selbstständig zum Ziel fährt. Dies sind nur zwei
Beispiele einer unendlich langen Liste. Hinter diesen technischen Wundern stecken meist
mehrere Tausend kleiner Mikrocomputern und Mikrocontrollern, die dann mittels Software zusammen
interagieren. Die Kombination dieser zwei Komponenten werden durch den Oberbegriff „Embedded
System“ oder auch zu Deutsch ein „Eingebettetes System“ definiert.
\newline
\newline
Embedded Systems können dabei grundsätzlich zwischen zwei Plattformen unterschieden werden:
\begin{itemize}
    \item Deeply Embedded System
    \item Non-Deeply Embedded System
\end{itemize}
\newline
\newline
Deeply Embedded Systems sind die wesentlichen Bausteine des Internet of Things
\cite{HochschuleniederrheimDeeply}. Die Anwendungen,
die bei Deeply Embedded Systems implementiert wird, basiert auf speziell angepassten
Echtzeitbetriebssystemen, den Programmiersprachen C oder C++ und ganz speziellen
\ac{gui}-Frameworks wie zum Beispiel TouchGFX.
\newline
\newline
Anders als bei den Deeply Embedded Systems, die sehr auf speziellen Technologien aufbauen, bieten
Open Embedded Systems eine höhere Flexibilität in Sachen Technologien an. Von Seiten der
Programmierung ist es möglich, unterschiedliche Technologien und Programmiersprachen zu verwenden.
Dort gilt bis dato unter Linux, die Kombination von C++ und Qt für\ac{gui}-lastige Systeme als
„State of the Art“.
\newline
\newline
Die Kombination aus C++ und Qt hat bislang auch funktioniert. Jedoch kommt dieser Ansatz
auch mit Herausforderungen mit sich. Nicht nur die höheren Entwicklungszeiten für die
Entwicklung von C++ Anwendungen, sondern auch die geringe Verfügbarkeit von Experten auf dem
Arbeitsmarkt sorgen für schlechtere Softwarequalität und längere Produktionszeiten.
\newline
\newline
Um diesen Herausforderungen zu entgegnen, soll in dieser Abschlussarbeit ein anderer Ansatz
betrachtet
werden. Und zwar könnten sowohl die Anwendungsschicht als auch die Persistenzschicht als .Net
Core Anwendungen implementiert werden. Als \ac{gui}-Technologie soll dabei die neue
Microsoft-Technologie namens \emph{Blazor} als Qt Ersatz zum Einsatz kommen. Somit kann erreicht
werden, dass die komplette Applikation in .Net Core implementiert werden kann.
\section{Aufgabenstellung}
\label{sec:aufgabenstellung}
Das Ziel dieser Arbeit ist die Entwicklung eines Blazor-basierten Frontend auf einem Raspberry Pi
4B um einen aussagekräftigen Vergleich zwischen den Technologien schaffen zu können und um eine
mögliche verdrängung mittels Blazor zu demonstrieren. Dazu soll
zunächst begutachtet werden, wie der momentane Stand der Technik für Open Embedded Systems ist,
um anschließend die Codebasis auf .Net Core und Blazor zu wechseln. Insbesondere sollen dabei
verschiedene Aspekte, wie zum Beispiel das Verhalten zur Laufzeit, dieses Ansatzes überprüft werden.
\begin{figure}[h]
    \centering
    \includegraphics[width=0.6\textwidth, center]{Einleitung/blazorxraspberry}
    \caption[Blazor mit Raspberry Pi]{Blazor mit Raspberry Pi}
    \label{img:blazorxraspberry}
\end{figure}
\section{Verwendet Hardware}
\label{sec:verwendeteHardware}
Als Zielplattform für diese Thesis dient ein Raspberry Pi 4 B. Dieses wird unter anderem deswegen
verwendet, da es im Labor Embedded Systems 2 der Hochschule Offenburg verwendet wird, aber auch,
da es sehr gut im Open Embedded Systems bereich eingesetzt werden kann. Der Raspberry Pi 4 B
verfügt dabei, unter anderem, über die folgenden technischen Spezifikationen: \cite{RasberryPiSpecs}
\begin{itemize}
    \item 1,5 GHz ARM Cortex-A72 Quad-Core-CPU
    \item 1 GB, 2 GB oder 4 GB LPDDR4 SDRAM
    \item Gigabit LAN RJ45 (bis zu 1000 Mbit)
    \item Bluetooth 5.0
    \item 2x USB 2.0 / 2x USB 3.0
    \item 2x microHDMI (1x 4k @60fps oder 2x 4k @30fps)
    \item 5V/3A @ USB Typ-C
    \item 40 GPIO Pins
    \item Mikro SD-Karten slot
\end{itemize}
Und wird mit dem \emph{Raspbian Buster with desktop} auf der SD-Karte betrieben.
\newline
Um noch mehr Funktionalität aufbringen zu können, wurde auf dem Raspberry Pi ein
\emph{RPI SENSE HAT Shield} aufgesteckt. Mit hilfe des \emph{RPI SENSE HAT Shield} können dann
unter anderem Daten wie zum Beispiel die momentane Temperatur oder auch die momentane
Luftfeuchtigkeit gewonnen werden. Zudem ist auf dem SENSE HAT noch eine 8x8 LED-Matrix enthalten
und ein Joystick mit 5 knöpfen, die angesteuert werden können.
\begin{figure}[h]
    \centering
    \includegraphics[width=0.5\textwidth, center]{Einleitung/pi-sense}
    \caption[Raspberry Pi 4 B]{Verwendeter Raspberry Pi}
    \label{img:piSense}
\end{figure}


\section{Verwendete Software}
\label{sec:verwendeteSoftware}
Im Rahmen dieser Thesis werden die folgenden Softwaretools zur Entwicklung eingesetzt.
\begin{itemize}
    \item Um eine visualisierung des Images \emph{Raspbian Buster with desktop} von dem Raspberry
    Pi zu erhalten, wird die Windows Desktop Anwendung \emph{VNC Viewer} verwendet. Dadurch ist
    die möglichkeit gegeben, bequem und einfach den Raspberry Pi über einen Bildschrim zu bedienen.
    \item Für die demonstrierung des Kapitels \emph{\nameref{chap:standTechnik}} wird eine
    beispielhafte grafische Oberfläche mithilfe des Qt-Creators implementiert.
    \item Den Zugriff auf die Funktionalitäten des \emph{RPI SENSE HAT Shield}, wird mittels der
    \emph{RTIMUlib} Bibliothek realisiert.
    \item Eine Ausschlagebende Technologie dieser Thesis wird durch dass Framework \emph{Blazor}
    von Microsoft abgebildet.
    \item Die Programmiersprachen dieser Thesis werden sich hauptsächlich aus \emph{C++} und
    \emph{C\#} beziehungsweise .Net zusammensetzten.
    \item Um den Zugriff auf die Funktionalitäten des \emph{RPI SENSE HAT Shield} mittels .Net zu
    gewährleisten, wird die Iot Bibliothek von Microsoft verwendet.
    \item Die Implementierung der Blazor anwendung wird dann letztendlich mittels der kostenlosen
    IDE \emph{Visual Studio Code} realisiert.
\end{itemize}

Die oben vorgestellten Tools und Programmiersprachen wurden nicht expliziet vorgegeben, sind
dementsprechend auch selbst ausgesucht und sind zu beginn dieser Thesis auch schon alle komplett
eingerichtet.
