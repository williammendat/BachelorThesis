\section{Was ist Blazor?}
\label{subsec:wasIstBlazor}
Blazor ist ein Framework von Microsoft zum Erzeugen von Webseiten. Der Name \emph{Blazor}
entstand aus der Kombination der zwei Wörter \emph{Browser} und \emph{Razor}. Browser zum einen,
da das Ziel war, C\# in den Browser zu bekommen und Razor zum anderen, da Blazor gebrauch von der
Razor syntax macht \cite{HierKommtBlazor}[vgl.]. Es wurde 2019 erstmalig von Microsoft
veröffentlicht, mit der Intension Webseiten oder auch \ac{spa} mithilfe von C\# zu entwickeln.
Dabei existieren 2 Varianten von Blazor:

\begin{itemize}
    \item Blazor Server
    \item Blazor WebAssembly
\end{itemize}
Der essenzielle Unterschied der beiden varianten besteht darin, dass Blazor Server auf einem
Server gehostet wird und Blazor WebAssembly native im Browser läuft, dazu aber im späteren
verlauf dieser Thesis mehr \cite{WasIstBlazor}[vgl.].
\newline
\newline
Die Idee, die hinter dem Framework von Microsoft steckt, ist es nicht nur C\# in den Browser zu
bekommen, um somit Javascript zu verdrängen, sondern auch mit nur einer Codebasis sowohl im
Frontend, als auch im Backend zu entwickeln. Somit wurde geschaffen, dass langjährige C\#
Entwickler mit ihrem vorhandenen Wissen als Full-Stack entwickler, eingesetzt werden können.
