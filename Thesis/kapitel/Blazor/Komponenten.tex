\section{Komponenten}
\label{subsec:komponenten}
Anders als es bei Qt der Fall ist, basiert Blazor wie andere Web-Frameworks auf \emph{Html} und
\emph{Css}. Das bedeutet, dass der Entwickler auf jedes Html element zugreifen kann, das
existiert. Weitergehend hat der Entwickler noch die möglichkeit auf zusätzliche Infragistics wie
zum Beispiel \emph{MadBlazor} oder \emph{Ignite UI} zurückzugreifen.
\newline
\newline
Außerdem bietet Blazor zusätzlich noch die Möglichkeit eigene Komponenten zu erstellen. Eine
Komponente, im Sinne von Web-Frameworks, kann sich so wie eine Methode in einer Programmiersprache
vorgestellt werden, das heißt, es kann eine Html und Css Sequenz ausgelagert werden, um diese an
mehrere Stellen zu verwenden. Hinzukomment kann eine Komponente, in Blazor, in zwei Varianten
vorkommen. Einmal als eine \emph{Page-Komponente} und einer \emph{Non-Page-Komponente}. Der
Unterschied zwischen den beiden ist, dass eine Page-Komponente durch eine URL adressieren
kann, das heißt, dass diese Komponente als eigenständige Seite einer Webseite angesehen werden
kann und eine Non-Page-Komponenten, wirklich nur die Sequenz von Html und Css darstellt
\cite{BlazorComponents}[vgl.].

\lstinputlisting[language={[Sharp]C},caption={Page-Komponente Beispiel},
    label=lst:counterpage]{\srcloc/Blazor/Counter.cs}

Listing \ref{lst:counterpage} zeigt eine Page-Komponete, die sowohl einen Text und einen Button
anzeigt und den momentane wert von \emph{currentCount} um eins erhöht wenn der Button betätigt
wird. Bei der Komponente handelt es sich um eine Page-Komponeten, da dises mit dem \emph{@page}
in der ersten Zeile ausgezeichnet ist.
\newline
\newline
Einer Komponente, sei es nun eine Page-Komponente oder Non-Page-Komponete, können auch einen oder
mehrere Parameter von der Oberkomponente mitgegeben werden. Dies hat den Vorteil, dass
Komponenten mehr an flexibilität gewinnen und somit besser wiederzuverwenden sind, wie im
folgenden Beispiel zu sehen ist:
\lstinputlisting[language={[Sharp]C},caption={Kind Komponente},
    label=lst:childkomponente]{\srcloc/Blazor/Child.razor}

\lstinputlisting[language={[Sharp]C},caption={Eltern Komponente},
    label=lst:parentkomponente]{\srcloc/Blazor/Parent.razor}

Wie in Listing \ref{lst:childkomponente} zu sehen ist, hat diese Komponente einen Parameter, die
sie dann, wie hier im Beispiel, als Title in einem \emph{<h2>-Tag} darstellt. Listig
\ref{lst:parentkomponente} kann dann gebrauch von der Kind Komponente machen und ihr einen Title
mitgeben. Der Html Tag der Kind Komponente, wird durch den Dateinamen der Komponente erzeugt, das
bedeutet, da die Komponente \emph{Child.razor} im Projekt heißt, wird diese dann auch mittels dem
\emph{<Child>-Tag} verwendet.
