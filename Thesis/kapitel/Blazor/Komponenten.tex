\section{Komponenten}
\label{subsec:komponenten}
Anders als bei Qt der Fall, basiert Blazor wie andere Web-Frameworks auf \emph{HTML} und
\emph{CSS}. Das bedeutet, dass Entwickler*innen auf jedes HTML-Element zugreifen können.
Weitergehend haben Entwickler*innen noch die Möglichkeit auf zusätzliche
Komponentenanbieter wie zum Beispiel \emph{MadBlazor} oder \emph{Ignite UI} zurückzugreifen.
\newline
\newline
Außerdem bietet Blazor zusätzlich die Möglichkeit eigene Komponenten zu erstellen. Eine
Komponente ist ein eigenständiger Teil der Benutzeroberfläche \cite{Komponenten}[vgl.].
Hinzukommend kann eine Blazor-Komponente, in zwei Varianten
vorkommen. Einmal als eine \emph{Page-Komponente} und eine \emph{Non-Page-Komponente}.
\begin{itemize}
    \item \emph{Page-Komponente} kann über die URL adressiert werden
    \item \emph{Non-Page-Komponente} kann nicht adressiert werden \cite{Komponenten}[vgl.]
\end{itemize}

\lstinputlisting[language={[Sharp]C},caption={Beispiel einer Page-Komponente},
    label=lst:counterpage]{\srcloc/Blazor/Counter.cs}

Listing \ref{lst:counterpage} zeigt eine Page-Komponete. Bei der Komponente handelt es sich um
eine Page-Komponente, da diese mit \emph{@page}
in der ersten Zeile ausgezeichnet ist.
\newline
\newline
Einer Komponente kann auch ein oder
mehrere Parameter von der Oberkomponente mitgegeben werden. Dies hat den Vorteil, dass
Komponenten mehr an Flexibilität gewinnen und somit besser wiederzuverwenden sind. Wie im
folgenden Beispiel zu sehen ist:
\lstinputlisting[language={[Sharp]C},caption={Kindkomponente},
    label=lst:childkomponente]{\srcloc/Blazor/Child.razor}

\lstinputlisting[language={[Sharp]C},caption={Elternkomponente},
    label=lst:parentkomponente]{\srcloc/Blazor/Parent.razor}

Wie in Listing \ref{lst:childkomponente} zu sehen ist, bekommt die Kindkomponente ihren Title als
Parameter übergeben. Listing
\ref{lst:parentkomponente} kann mithilfe des Title-Attributes den Title an die Kindkomponente
übergeben. Der HTML-Tag der Kindkomponente, wird durch den Dateinamen der Komponente erzeugt. Da
die Komponente \emph{Child.razor} im Projekt heißt, wird diese mittels dem
\emph{<Child>-Tag} verwendet.
