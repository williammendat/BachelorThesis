\subsection{Javascript Invokable}
\label{subsec:jsInvokable}
Nun kann es auch den Fall geben, dass C\# code von Javascript aufgerufen werden muss. Dies ist in
Blazor ebenso möglich und kann vollbracht werden, mit dem JSInvokable Attribut, die über einer
Methode angebracht werden muss. Zudem muss die Methode als \emph{static} implementiert werden.
\newline
\newline
Auf Javascript seite, wird von Blazor die Methoden
\begin{itemize}
    \item DotNet.invokeMethod (string, string, object[])
    \item DotNet.invokeMethodAsync (string, string, object[])
\end{itemize}
bereitgestellt, mit denen dann eine statische C\# Methode aufgerufen werden kann. Dabei gibt der
Entwickler mithilfe des ersten Parameters, den Project Namen an, in der sich die statische
Methode befindet, der zweite Parameter, ist für den Namen der aufzurufenden Methode gedacht und
zu guter letzt können mit dem dritten Parameter noch weitere optionale Argumente übergeben werden.
\newline
\newline
Im folgenden, wird eine Komponente gezeigt welche beim Betätigen eines Buttons, eine Javascript
Methode aufruft, welche wiederum einen generierten string von einer C\# Methode erhält und diese
in einem \emph{alert} ausgibt:

\lstinputlisting[language={[Sharp]C},caption={Javascript Invokable Beispiel},
    label=lst:jsInvokableExample]{\srcloc/Blazor/JsInvokable.razor}

\lstinputlisting[language=JavaScript,caption={Javascript Invokable Show Generated Message},
    label=lst:jsInvokable]{\srcloc/Blazor/testInvokable.js}