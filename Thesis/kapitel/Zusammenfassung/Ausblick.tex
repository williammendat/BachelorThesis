\chapter{Ausblick}
\label{chp:ausblick}
Abschließend soll noch ein kurzer Ausblick über die weitere Entwicklung von Benutzeroberflächen für
Non-Deeply Embedded Systems gegeben werden. Die beiden vorgestellten Frameworks dieser Arbeit
eignen sich beide für die Entwicklung einer Benutzeroberfläche. Jedoch muss hier differenziert
werden, welche Kriterien die Benutzeroberfläche erfüllen muss.

In dem Fall, dass die Performance sehr wichtig ist, sollte auf das Framework Qt in
Kombination mit C++ zurückgegriffen werden. Sollte jedoch ein schneller Entwicklungsprozess
vonnöten sein, ist das Framework Blazor definitiv eine gute Alternative.

In dieser Arbeit wurde zudem noch die Blazor Maui-Architektur vorgestellt, die aufgrund der nicht
vorhandenen Kompatibilität mit Linux, nicht weiter betrachtet werden konnte. Hier bleibt es
abzuwarten, ob Microsoft Linux mit \emph{Blazor} unterstützten wird. Sollte Linux in Zukunft
von Blazor Maui profitieren, so sollte eine solche Analyse erneut durchgeführt werden.