\chapter{Ausblick}
\label{chp:ausblick}
Abschließend soll noch ein kurzer Ausblick über die weitere Entwicklung für eine GUI für ein
Non-Deeply Embedded System gegeben werden. Die beiden vorgestellten Frameworks dieser Thesis
eignen sich beide für die Entwicklung einer Benutzeroberfläche. Jedoch muss hier differenziert
werden, welche kriterien die Benutzeroberfläche erfüllen muss.

In dem Fall, dass die Performance sehr wichtig ist, dann sollte auf das Framework Qt in
kombination mit C++ zurückgegriffen werden. Sollte jedoch ein schneller Entwicklungsprozess
vonnöten sein, dann wäre das Framework Blazor definitive eine gute alternative.

In dieser Arbeit wurde zudem noch die Blazor Maui-Architektur vorgestellt, die aufgrund der nicht
vorhandenen kompatibilität mit Linux nicht weiter betrachtet werden konnte. Hier bleibt es
abzuwarten, ob Microsoft damit auch noch Linux unterstützten wird. Sollte Linux in ferner Zukunft
von Blazor Maui profitieren, so sollte eine solche Analyse noch mal durchgeführt werden.