\chapter{Fazit}
\label{chp:fazit}
In dieser Abschlussarbeit wurde veranschaulicht, wie das Framework \emph{Blazor} genutzt werden
kann, um eine Benutzeroberfläche für ein Non-Deeply Embedded System zu erstellen. Es konnte
gezeigt werden, welche Bibliothek benutzt werden kann, um auf die Funktionalitäten eines
Non-Deeply Embedded System zuzugreifen. In dem Entwickelten Frontend, wurden kontinuierlich Daten
vom Raspberry Pi abgefragt, um diese dann auf der Benutzeroberfläche anzuzeigen. Im Anhang
\ref{lst:DemoCode} befindet sich der komplett implementierte Code.

Zusätzlich wurde im Umfang dieser Abschlussarbeit eine Laufzeitanalyse zwischen Qt und den beiden
Blazor-Architekturen durchgeführt. Bei dieser Analyse stellte sich dann herraus, dass Qt in den
meisten Fällen sich als performantesten gezeigt hat. Zudem zeigte sich ein interessantes
Verhalten zwischen den beiden Blazor-Architekturen. Bei geringen Datenmengen, schien die
Server-Architektur performanter als die WebAssembly-Architektur zu sein, jedoch bei großen
Datenmengen, war die WebAssembly-Architektur deutlich besser als die Server-Architektur.