\section{Qt}
\label{sec:qt}
In der vorherigen Sektion wurde ein allgemeines Bild dargestellt, was ein eingebettetes System
ist und in welchen Varianten diese auftauchen. Weitergehend soll nun, in dieser Sektion,
vorgestellt werden, mit dem üblicherweise im \emph{Open Embedded Systems} bereich programmiert wird.
\newline
\newline
Ein großer Anteil eines \emph{Open Embedded Systems} wird heutzutage durch die \ac{gui}
repräsentiert. Die \ac{gui} sollte intuitive und zuverlässig sein.
Zudem ist es auch noch enorm wichtig, dass die \ac{gui} wenige Resourcen verbraucht, schnell
reagiert und einfach einzubinden ist. Um diese Anforderung zu erreichen, wurde bis Dato \emph{Qt} in
Kombination mit der Programmiersprache \emph{C++} verwendet \cite{QtOnEmbeddedLinux}[vgl.].

\subsection{Was ist Qt?}
\label{subsec:WasIstQt}
Qt ist ein Framework zum Erzeugen von GUI´s auf mehreren Betriebssystemen. Es wurde 1990 von
\emph{Haarvard Nord} und \emph{Eirik Chambe-Eng} unter dem Vorwand entwickelt, benutzer
freundliche GUI´s mithilfe der Programmiersprache C++ zu entwickeln. Der Name \emph{Qt}
entstand dabei, da Haarvard den Buchstaben \emph{Q} in Emacs als sehr schön empfand. Zudem entstand das \emph{t} in Qt als
abkürzung für das englische Wort \emph{Toolkit} \cite{qtStory}[vgl.].
\newline
\newline
Die Intension hinter Qt war es damals nicht nur ein Framework zu erschaffen, welches benutzer
freundliche GUI´s kreiert, sondern dies auch, mit nur einer Code-Basis über alle Betriebssysteme
hinweg. Die schwierigkeit bestand also darin, das selbe Aussehen, die selbe benutzer Erfahrung
und die gleiche Funktionalität über die verschiedenen Betriebssysteme zu
schaffen \cite{GettingStartedQt}[vgl.].
\newline
\newline
Qt ist in C++ entwickelt worden und verwendet zusätzlich noch einen Compiler welcher
\emph{\ac{moc}} genannt wird, womit C++ um weitere Elemente erweitert wird. Der daraus
compilierte Code folgt dem C++-Standard und ist somit mit jedem anderen C++ Compiler kompatibel.
Obwohl Qt ursprünglich dafür gedacht war, rein auf C++ zu basieren, wurden im Laufe der Zeit
mehrere Eweiterungen für Qt von der Community entwickelt, um Qt mit mehreren Programmiersprachen
benutzen zu können, wie zum Beispiel Python oder Java.
\begin{figure}[h]
    \centering
    \includegraphics[width=0.4\textwidth, center]{StandDerTechnik/qtLogo}
    \caption[Qt Logo]{Qt Logo}
    \label{img:qtLogo}
\end{figure}
\subsection{Programmierbeispiel}
\label{subsec:programmierbeispiel}
Das Programmierbeispiel welches im Folgenden dargestellt wird, erzeugt ein Fenster mit dem Titel 
\emph{Meine erste Qt App}, ein Label welches \emph{Hello World} anzeigt und ein Button mit der
Aufschrift \emph{Exit}. Sobald der Button oder die Tastenkombination \emph{Alt + E} gedrückt
wird, schließt sich das Fenster.

\lstinputlisting[language=C++,caption={Qt Hello World Sourcecode},
    label=lst:qtHelloWorldSourceCode]{\srcloc/StandDerTechnik/qtHelloWorld.cpp}

Das Programm welches erzeugt wird, wird folgend dargestellt:
\begin{figure}[h]
    \centering
    \includegraphics[width=0.5\textwidth, center]{StandDerTechnik/qtHelloWorldApp1}
    \caption[Qt Hello World App]{Qt Hello World App}
    \label{img:qtHelloWorldApp}
\end{figure}
\subsection{Widgets}
\label{subsec:widgets}
Um eine \ac{gui} gestalten zu können braucht es Komponenten, die auf der \ac{gui} angezeigt
werden können. Qt verwendet dafür sogenannte \emph{Widgets}. Widgets sind als grafische
Komponenten, die dafür genutzt werden können, um die Benutzeroberfläche nach Belieben zu
gestalten. Ein beispiel für eine solche Komponente wäre ein Button, welcher in der Sektion
\emph{\nameref{subsec:programmierbeispiel}} als \emph{btnExit} vorkam.
\newline
\newline
Die Widgets, die Qt zur verfügung stellt sind in einer großen Klassenhierarchie zusammengesetzt
und diese Hierarchie könnte sich wie folgt vorgestellt werden:
\begin{figure}[h]
    \centering
    \includegraphics[width=\textwidth, center]{StandDerTechnik/qtWidgetClassHierachie}
    \caption[Qt Widgets Klassenhierachie]{Qt Widgets Klassenhierarchie
    \cite{GettingStartedQt}[vgl.]}
    \label{img:qtWidgetClassHierachie}
\end{figure}
Wie zu sehen ist, ist ganz oben in der Klassenhierarchie die \emph{QObject} Klasse. Diese
enthällt unter anderem den Signal und Slot mechanismus, der Später noch genau erklärt wird.
Weitergehend, werden Widgets die gemeinsame Funktionalitäten aufweisen zusammen grupiert. Diese
Verhalten ist bei \emph{QPushButton} und \emph{QRadioButton} erkennbar, denn beide Widgets sind
Buttons, die sich Teilweise die gleichen Eigenschaften und Funktionen teilen
\cite{GettingStartedQt}[vgl.].

\subsection{Signal und Slot Konzept}
\label{subsec:signalslot}
Um eine Benutzeroberfläche wirklich interaktive zu gestalten, ist es vonnöten, auf bestimmte
Ereignisse zu reagieren. Beispielsweise ist das betätigen eines Buttons ein Auslöser für ein
Ereignis auf welches reagiert werden kann. Es gibt viele Methoden und Muster in welche ein
solches Ereignis-Aktions Konzept implementiert werden kann.

Qt benutzt um diese Ereignis-Aktion Muster zu bewerkstelligen, das selbst entwickelte Signal und
Slot Konzept. Ein \emph{Signal} ist im einfachen sinne eine Nachricht, die Versendet wird, um zu
signalisieren, dass der momentane Status eines Objektes sich geändert hat. Dahingegen ist ein
\emph{Slot} eine spezielle Funktion von einem Objekt, welche immer dann aufgerufen wird, wenn ein
bestimmtes \emph{Signal} gesendet wird \cite{GettingStartedQt}[vgl.].

Damit jeder \emph{Slot} auch weiß, auf welches \emph{Signal} reagiert werden soll, müssen diese
zusammen verbunden werden. In der Sektion \emph{\nameref{subsec:programmierbeispiel}} war
folgende Zeile zu sehen:

\begin{lstlisting}[language=C++, caption=Signal und Slot beispiel, label=lst:SignalSlotBeispiel]
// Connect button with the App
QObject::connect(btnExit, SIGNAL(clicked()), &app, SLOT(quit()));

\end{lstlisting}

In dieser Codezeile fand die Verbindung zwischen einem Signal und einem Slot statt. Dabei wurde
sich mit dem Button \emph{btnExit} verbunden und ein Signal gesendet, wenn der Button betätigt
wird. Weitergehend wurde sich noch mit \emph{app} verbunden, auf welchem dann, beim Betätigen von
btnExit, die Funktion \emph{quit} aufgerufen wird. Sobald also das Signal gesendet wird, schließt
sich die Application.

Einer der größten Vorteile dieser Methode im gegensatz zu anderen Methoden, ist es, dass dadurch
N zu M Beziehungen abgebildet werden können. Das bedeutet also, dass sich ein Signal mit
beliebig viele Slots verbinden kann und dass sich ein Slot auf mehrere Signale verbinden kann
\cite{SignalSlotMechanismus}[vgl.].
%\subsection{Raspberry Pi und Qt}
\label{subsec:rasPiUndQt}
test