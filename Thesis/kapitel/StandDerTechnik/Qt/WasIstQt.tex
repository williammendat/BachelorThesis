\subsection{Was ist Qt?}
\label{subsec:WasIstQt}
Qt ist ein Framework zum Erzeugen von GUI´s auf mehreren Betriebssystemen. Es wurde 1990 von
\emph{Haarvard Nord} und \emph{Eirik Chambe-Eng} unter dem Vorwand entwickelt, benutzer
freundliche GUI´s mithilfe von der Programmiersprache C++ zu entwickeln. Der Name \emph{Qt}
entstand dabei, da Haarvard den Buchstaben \emph{Q} in Emacs\footnote{Emacs ist ein
leistungsfähiger Texteditor und im Unterschied zu den meisten Editoren eine komplette
Arbeitsumgebung \cite{EmacsKurz}} als sehr schön empfand. Zudem entstand das \emph{t} in Qt als
abkürzung für das englische Wort \emph{Toolkit} \cite{qtStory}[vgl.].
\newline
\newline
Die Intension hinter Qt war es damals nicht nur ein Framework zu erschaffen, welches benutzer
freundliche GUI´s kreiert, sondern dies auch, mit nur einer Code-Basis über alle Betriebssysteme
hinweg. Die schwierigkeit bestand also darin, das selbe Aussehen, die selbe benutzer Erfahrung
und die gleiche Funktionalität über die verschiedenen Betriebssysteme zu
schaffen \cite{GettingStartedQt}[vgl.].
\newline
\newline
Qt ist in C++ entwickelt worden und verwendet zusätzlich noch einen Compiler\footnote{Ein
Compiler, im einfachen sinne, ist ein Übersetzter, der von einer Programmiersprache in
Computersprache (0 und 1) Übersetzt \cite{WasIstEinCompiler}[vgl.].} welcher
\emph{\ac{moc}} genannt wird, womit C++ um weitere Elemente erweitert wird. Der daraus
compilierte Code folgt dem C++-Standard und ist somit mit jeden anderen C++ Compiler kompatibel.
Obwohl Qt ursprünglich dafür gedacht war, rein auf C++ zu basieren, wurden im Laufe der Zeit
mehrere Eweiterungen für Qt von der Community entwickelt, um Qt mit mehreren Programmiersprachen
benutzen zu können, wie zum Beispiel Python oder Java.
\begin{figure}[h]
    \centering
    \includegraphics[width=0.25\textwidth, center]{StandDerTechnik/qtLogo}
    \caption[Qt Logo]{Qt Logo}
    \label{img:qtLogo}
\end{figure}