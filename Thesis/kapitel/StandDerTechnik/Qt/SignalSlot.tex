\subsection{Signal und Slot Konzept}
\label{subsec:signalslot}
Um eine Benutzeroberfläche wirklich interaktive zu gestalten, ist es vonnöten, auf bestimmte
Ereignisse zu reagieren. Beispielsweise ist das betätigen eines Buttons ein Auslöser für ein
Ereignis auf welches reagiert werden kann. Es gibt viele Methoden und Muster in welche ein
solches Ereignis-Aktions Konzept implementiert werden kann.

Qt benutzt um diese Ereignis-Aktion Muster zu bewerkstelligen, das selbst entwickelte Signal und
Slot Konzept. Ein \emph{Signal} ist im einfachen sinne eine Nachricht, die Versendet wird, um zu
signalisieren, dass der momentane Status eines Objektes sich geändert hat. Dahingegen ist ein
\emph{Slot} eine spezielle Funktion von einem Objekt, welche immer dann aufgerufen wird, wenn ein
bestimmtes \emph{Signal} gesendet wird \cite{GettingStartedQt}[vgl.].

Damit jeder \emph{Slot} auch weiß, auf welches \emph{Signal} reagiert werden soll, müssen diese
zusammen verbunden werden. In der Sektion \emph{\nameref{subsec:programmierbeispiel}} war
folgende Zeile zu sehen:

\begin{lstlisting}[language=C++, caption=Signal und Slot beispiel, label=lst:SignalSlotBeispiel]
// Connect button with the App
QObject::connect(btnExit, SIGNAL(clicked()), &app, SLOT(quit()));

\end{lstlisting}

In dieser Codezeile fand die Verbindung zwischen einem Signal und einem Slot statt. Dabei wurde
sich mit dem Button \emph{btnExit} verbunden und ein Signal gesendet, wenn der Button betätigt
wird. Weitergehend wurde sich noch mit \emph{app} verbunden, auf welchem dann, beim Betätigen von
btnExit, die Funktion \emph{quit} aufgerufen wird. Sobald also das Signal gesendet wird, schließt
sich die Application.

Einer der größten Vorteile dieser Methode im gegensatz zu anderen Methoden, ist es, dass dadurch
N zu M Beziehungen abgebildet werden können. Das bedeutet also, dass sich ein Signal mit
beliebig viele Slots verbinden kann und dass sich ein Slot auf mehrere Signale verbinden kann
\cite{SignalSlotMechanismus}[vgl.].