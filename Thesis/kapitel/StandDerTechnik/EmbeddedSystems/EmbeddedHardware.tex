\subsection{Hardware}
\label{subsec:EmbeddedHardware}
Die einzelnen Komponenten, die in einem Embedded Systems verbaut worden sind, entscheiden über
die vorhandene Leistung, den Stromverbrauch und die Robustheit, die dieses System ausmachen. Die
kern Komponente eines Embedded System wird durch einen Prozessor repräsentiert und
wird häufig als \emph{System on Chip} eingebaut. Dabei ist die Tendenz zu den ARM-Core Modellen
steigend. Neben dem Prozessor befinden sich typischerweise auf den Embedded Systems noch
weitere Komponenten, wie zum Beispiel dem Hauptspeicher, dem persistenten Speicher und diversen
Schnittstellen, um weitere peripherie Geräte anzuschließen \cite{EmbeddedLinuxQuade}[vgl.].
\newline
\newline
Damit weitere peripherie Geräte angeschlossen werden können, müssen mittels einigen Leitungen,
digitale Signale übertragen werden. Aufgrund dessen, dass Leitungen typischerweise nur über eine
begrenzte Leistung verfügen, werden Treiber eingesetzt, um peripherie Geräte zu verbinden.
Nicht selten kommt es vor, dass in einem Embedded System darüber hinaus noch eine galvanische
Entkopplung eingebaut wird, damit die Hardware
vor Störungen von außen, wie Beispiel Motoren, geschützt ist \cite{EmbeddedLinuxQuade}[vgl.].