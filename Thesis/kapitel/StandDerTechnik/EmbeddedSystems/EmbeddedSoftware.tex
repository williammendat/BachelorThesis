\subsection{Software}
\label{subsec:EmbeddedSoftware}
Genauso, wie sich Embedded Systems in zwei Bereiche unterscheiden lassen können, kann die
Software eines Embedded System in Systemsoftware und funktionsbestimmende
Anwendungssoftware unterteilt werden. Da \emph{Deeply Embedded Systems} meinst nur für kleine und
einfach Aufgaben ausgelegt sind, benötigen diese meist nur sehr schwache Software. Diese
funktionsbestimmende Anwendungssoftware ist dann meist ein spezielles Echtzeitbetriebssystem,
welches auf die Aufgabe und der Hardware angepasst ist.
\newline
\newline
Ganz anders sieht es im \emph{Open Embedded Systems} bereich aus, welche seine Software als
Kennzeichen hat. Da diese für sehr komplexe Aufgaben zum einsatzt kommen, kommt es nicht selten
vor, dass auch eine \ac{gui} für ein solches System vonnöten ist. Deswegen basiert ein \emph{Open
Embedded Systems} auf einer Systemsoftware, die dem Programmierer mehr Möglichkeiten beim
Entwickeln gibt. Unter anderem ist somit auch die Möglichkeit gegeben, die Programmiersprache und
das \ac{gui}-Framework, nach Belieben selbst auszusuchen und nicht auf spezielle Software angewiesen
zu sein \cite{EmbeddedLinuxQuade}[vgl.].