\subsection{Software}
\label{subsec:EmbeddedSoftware}
Genauso wie sich Embedded Systems in zwei Bereiche unterscheiden lassen können, kann die
Software eines Embedded System in Systemsoftware und funktionsbestimmende
Anwendungssoftware unterteilt werden.
Für ein \emph{Deeply Embedded System} wird in den meisten fällen ein Echtzeitbetriebssystem
(Realtime Operating System - RTOS) verwendet, welches an die Hardware angepasst ist.
\newline
\newline
Ganz anders sieht es im \emph{Non-Deeply Embedded Systems} Bereich aus. Da diese für sehr
komplexe Aufgaben zum Einsatz kommen, kommt es nicht selten
vor, dass eine \ac{gui} für ein solches System vonnöten ist. Deswegen basiert ein
\emph{Non-Deeply
Embedded Systems} auf einer Systemsoftware, die Programmierer*innen mehr Möglichkeiten bei
der Entwicklung geben. Unter anderem ist die Möglichkeit gegeben, die Programmiersprache
und
das \ac{gui}-Framework selbst auszusuchen \cite{EmbeddedLinuxQuade}[vgl.].